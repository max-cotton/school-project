\documentclass[./project-report/src/latex/project-report.tex]{subfiles}

\begin{document}

\maketitle

\section{Evaluation TODO}

\subsection{Project Objectives Evaluation}

\subsubsection{Project Objectives}

For the reader's convenience, I have restated the project objectives below:
\vspace{5mm}

\noindent\begin{tabular}{|p{0.13\linewidth}|p{0.87\linewidth}|}
    \hline
    \textbf{Objective ID} & \textbf{Description} \\
    \hline
    1 & Learn how Artificial Neural Networks work and develop them from first principles \\
    \hline
    2 & Implement the Artificial Neural Networks by creating trained models on image datasets \\
    \hline
    2.1 & Allow use of Graphics Cards for faster training \\
    \hline
    2.2 & Allow for the saving and loading of trained models \\
    \hline
    3 & Develop a Graphical User Interface \\
    \hline
    3.1 & Provide controls for hyper-parameters of models \\
    \hline
    3.2 & Display and compare the results each model's predictions \\
    \hline
\end{tabular}

\subsubsection{Project Objective Evaluations}

\begin{tabular}{|p{0.13\linewidth}|p{0.59\linewidth}|p{0.12\linewidth}|p{0.16\linewidth}|}
    \hline
    \textbf{Objective ID} & \textbf{Evaluation} & \textbf{Status} & \textbf{3rd Party Evaluation} \\
    \hline
    1 & I have learnt how Artificial Neural Networks work from online resources, reports and interviewing a subject matter expert. I have proven the key mathematical principles from first principles 
        and implemented these structures within Python code. & Fully met & Fully met \\
    \hline
    2 & I have implemented trainable Artificial Neural Networks with configurable numbers of layers, number of neurons in each layer and the nature of the Transfer Functions. The Artificial Neural 
        Networks have been trained and tested on a variety of datasets and operates at an accuracy level comparable with the resources learnt from. & Fully met & Fully met \\
    \hline
    2.1 & The Artificial Neural Networks allow the use of a graphics card where applicable. & Fully met & Fully met \\
    \hline
    2.2 & The trained Artificial Neural Networks' weights and biases can be saved to a data file and the features of the corresponding Artificial Neural Networks are saved to a database. These saved 
          Artificial Neural Networks can be loaded independently. & Fully met & Fully met \\
    \hline
    3 & A Graphical User Interface allowing configuration of all hyper-parameters, loading and saving of trained models and testing has been developed. & Fully met & Fully met \\
    \hline
    3.1 & The Graphical User Interface allows user configuration of all utilised model hyper-parameters. & Fully met & Fully met \\
    \hline
    3.2 & The model predictions can be compared in terms of both learning rate and overall accuracy. & Fully met & Fully met \\
    \hline
\end{tabular}

\subsection{Third Party Feedback}

I demonstrated the final version of my program to the same third party that I interviewed in the analysis, and their response is shown below:

\vspace{5mm}

"In my opinion, Max has definitely met the primary and secondary goals of this project. Firstly, and most importantly, he has researched and implemented, from first 
principles, an Artificial Neural Network that is flexible and abstracted to the point that it can tackle a range of problems. Max started the analysis for this 
project from a very theoretical and mathematical point of view before implementing code which has allowed him to extend its implementation to a range of datasets 
from the XOR problem to image analysis.

I was particularly impressed at the level of analysis he undertook into how Artificial Neural Networks work and the impact that different kinds of design decisions can have on 
implementation. He took on board suggestions to explore different types of transfer functions such as the ReLu function to increase the speed of training and it was nice 
to see comparative studies of this and other techniques. It was also great to see the ability to save and load trained models which has allowed him to train models 
on a desktop PC equipped with a graphics card to be utilized on a lower power laptop.

The analysis section exploring the impact on both learning rates and epoch count was very nice to see, as well as the identification of an optimal learning rate 
suitable for the image dataset he was working with.

In summary, it was fantastic to see a true maths-to-code example of implementing Artificial Neural Networks that didn't rely on the use of external AI libraries. I 
am certain he has learned a great deal regarding the fundamental properties and limitations of Artificial Neural Networks. I was also impressed by the usage of 
software engineering tools such as GitHub and Jupyter Notebook throughout the project."

\subsection{Future Improvements}

By taking into consideration my evaluation of the objectives and feedback from the third party, I believe that the following future improvements could be made for the 
project:

\begin{itemize}
    \item For the analysis of the effects of hyper-parameters, repeated tests seeded with different parts of the training dataset could provide a more accurate 
          analysis of the average behaviour of the Artificial Neural Networks.
    \item Performing data augmentation to expand the size of the training datasets, such as by shifting, rotating, cropping and zooming into training images or by 
          adding noise to training images to produce more.
    \item Exploring Convolutional Neural Networks.
    \item Utilising a standardized file format for storing trained Artificial Neural Networks, so that they can be integrated with other machine learning libraries.
\end{itemize}

\end{document}