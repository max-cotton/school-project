\documentclass[10pt,a4paper]{article}
\usepackage[utf8]{inputenc}
\usepackage{amsmath}
\usepackage{amsfonts}
\usepackage{amssymb}
\usepackage{natbib}
\usepackage{graphicx}

\title{NumPy theory}
\author{Max Cotton}
\date{}

\begin{document}

\maketitle

\section{Array}

\begin{itemize}
    \item Use numpy arrays for numpy processes
    \item The "shape" attribute of a numpy array returns a tuple of the shape of the array, with the first index being the number of arrays in the 2D array and the second being the number of items in each array
    \item The "T" attribute of a numpy array shows the transpose of the array (How it acts as a matrice)
\end{itemize}

\section{Functions}

\begin{itemize}
    \item numpy.random.rand(d0,d1) generates a numpy array with a shape of (d0,d1) with random values
    \item numpy.expr(-z) returns $e^{-z}$
    \item numpy.log() is the natural logarithm (e is the base)
    \item numpy.dot(array1,array2) returns the dot product of two arrays
    \item numpy.reshape(array, shape) reshapes an array without changing its contents
    \item numpy.sum(array) returns the sum of all elements in the array
    \item numpy.squeeze() returns an array with the same data but reshaped so dimensions of length one are removed
\end{itemize}

\end{document}