\documentclass[10pt,a4paper]{article}
\usepackage[utf8]{inputenc}
\usepackage{amsmath}
\usepackage{amsfonts}
\usepackage{amssymb}
\usepackage{natbib}
\usepackage{graphicx}

\title{Cat Image Recognition model theory}
\author{Max Cotton}
\date{}

\begin{document}

\maketitle

\section{Setup}

\begin{itemize}
    \item Load image datasets from '.h5' files. (Cat image datasets sourced from https://github.com/marcopeix)
    \begin{itemize}
        \item Where each image in the input dataset, is represented by an R, G and B matrice, to store the RGB values of each pixel in the 64x64 pixel images
        \item Each image's RGB matrices are then 'flattened' into a 1 dimensional array of values, where each element is also divided by 255 (max RGB value) to a number between 0 and 1, to standardize the dataset
        \item The output dataset is also loaded, and is reshaped into a 1 dimensional array of 1s and 0s, to store the output of each image (1 for cat, 0 for non cat)
    \end{itemize}
    \item There is a training dataset with 209 pictures and a test dataset with 50 pictures (fewer pictures are needed for testing)
    \item Afterwards, the weights and the bias are all initialised to zero/s
\end{itemize}

\section{Model}
This cat image recognition model uses a Perceptron Artifical Neural Network model, with the RGB values as the input array, and uses the sigmoid transfer function to obtain a single output neuron/prediction between 0 and 1 (where a prediction greater than or equal to 0.5 predicts cat), for a binary classification of 'cat' or 'not a cat'

\end{document}